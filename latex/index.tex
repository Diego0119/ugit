\chapter{Tarea 1 estructura de datos (UGIT)}
\hypertarget{index}{}\label{index}\index{Tarea 1 estructura de datos (UGIT)@{Tarea 1 estructura de datos (UGIT)}}
\label{index_md__r_e_a_d_m_e}%
\Hypertarget{index_md__r_e_a_d_m_e}%
 u\+Git (ugit) 🛠️

u\+Git (ugit) es un simulador de un sistema de control de versiones que emula funcionalidades básicas de Git. Este proyecto está diseñado para proporcionar una experiencia práctica en la manipulación de estructuras de datos como arreglos y tablas hash en el lenguaje C.

Objetivos 🎯


\begin{DoxyItemize}
\item Comprender el concepto de control de versiones y cómo los sistemas como Git gestionan diferentes versiones de un proyecto.
\item Implementar y manipular estructuras de datos como arreglos y tablas hash para almacenar y gestionar eficientemente la información versionada.
\item Desarrollar habilidades en programación en C, centrándose en el manejo de memoria, punteros, y eficiencia algorítmica.
\item Simular operaciones de un sistema de control de versiones para obtener una comprensión más profunda de su funcionamiento interno.
\end{DoxyItemize}

Descripción del Proyecto 📚

u\+Git permite realizar operaciones básicas de control de versiones como\+:
\begin{DoxyItemize}
\item Agregar archivos
\item Crear commits
\item Ver el historial de versiones
\item Cambiar entre versiones
\end{DoxyItemize}

Estas funcionalidades se implementan utilizando estructuras de datos básicas en C. La finalidad del proyecto es proporcionar a los estudiantes una experiencia práctica en la implementación de conceptos fundamentales de control de versiones.

Normativas de Codificación en C 🧩

-\/Consistencia en el estilo de código\+:
\begin{DoxyItemize}
\item Usar nombres de variables y funciones descriptivos.
\item Mantener una indentación de 4 espacios por nivel. -\/Modularidad\+:
\item Dividir el código en funciones pequeñas y manejables.
\item Evitar funciones que superen las 50 líneas de código. -\/Documentación\+:
\item Identificar cada código con el programador.
\item Incluir comentarios claros y precisos.
\item Documentar cada función con propósito, parámetros de entrada, valores de retorno y posibles errores.
\end{DoxyItemize}

Gestión de errores\+:
\begin{DoxyItemize}
\item Implementar mecanismos robustos para capturar y manejar errores. -\/Uso eficiente de memoria\+:
\item Evitar el uso innecesario de variables globales.
\item Liberar memoria dinámica cuando ya no sea necesaria. -\/Optimización\+:
\item Escribir código optimizado en términos de tiempo y espacio. 
\end{DoxyItemize}